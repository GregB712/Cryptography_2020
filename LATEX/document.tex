%credits K. Κουνής, I. Στεφανίδου 
%refactoring : K.Draziotis (drazioti@gmail.com)
%Licence CC_BY_SA

\documentclass[12pt]{article}
\usepackage{helvet} 
\usepackage{amsmath}
\usepackage{titlesec}
\usepackage{lipsum}
\usepackage{graphicx}
\graphicspath{{images/}}


\usepackage{listings}
\usepackage{xcolor}

\definecolor{codegreen}{rgb}{0,0.6,0}
\definecolor{codegray}{rgb}{0.5,0.5,0.5}
\definecolor{codepurple}{rgb}{0.58,0,0.82}
\definecolor{backcolour}{rgb}{0.95,0.95,0.92}

\lstdefinestyle{mystyle}{
	backgroundcolor=\color{backcolour},   
	commentstyle=\color{codegreen},
	keywordstyle=\color{magenta},
	numberstyle=\tiny\color{codegray},
	stringstyle=\color{codepurple},
	basicstyle=\ttfamily\footnotesize,
	breakatwhitespace=false,         
	breaklines=true,                 
	captionpos=b,                    
	keepspaces=true,                 
	numbers=left,                    
	numbersep=5pt,                  
	showspaces=false,                
	showstringspaces=false,
	showtabs=false,                  
	tabsize=2
}
\lstset{style=mystyle}

\titleformat{\section}[display]
{\clearpage\vspace*{50pt}%
	\normalfont\huge\bfseries}%
{{\Kappa}E{\Phi}A{\Lambda}AIO \thesection}%
{20pt}%
{\Huge}%
[\vspace{40pt}]

\usepackage[algosection,commentsnumbered,ruled,vlined]{algorithm2e}
\NoCaptionOfAlgo
\usepackage{chemarrow}
\newcommand\aug{\fboxsep=-\fboxrule\!\!\!\fbox{\strut}\!\!\!}
\usepackage{graphicx}
\usepackage{gfsdidot}
\usepackage[LGR,T1]{fontenc}
\usepackage[utf8]{inputenc}
\usepackage[english,greek]{babel} % και για τις δυο γλώσσες
\usepackage{alphabeta}
\usepackage[hidelinks]{hyperref}
\usepackage{hyperref}
\usepackage{makeidx}
\usepackage{enumerate}
\usepackage{enumitem}
\usepackage{systeme}
\usepackage{algorithmic}
\usepackage{comment}

% TEXT FORMATTING
% set spacing between lines (διάστιχο)
\usepackage{setspace}
\setstretch{1.5}
% package to customize chapters, sections and subsections style
\usepackage{titlesec}
% chapter title appearance format
\titleformat{\chapter}[display]
{\bfseries\huge}{\chaptertitlename\space\thechapter}{16pt}{}
% https://www.sharelatex.com/learn/Sections_and_chapters
\titlespacing{\chapter}{0pc}{1.5ex plus .1ex minus .2ex}{5pc}
% section title appearance format
\titleformat{\section}
{\bfseries\large}{\thesection}{14pt}{}
% subsection title appearance format
\titleformat{\subsection}
{\bfseries\normalsize}{\thesubsection}{12pt}{}
% set margins
\usepackage{geometry}
\geometry{left=3cm, right=2cm, top=4cm, bottom=3cm}
\usepackage{graphicx}
% put images in images path
\graphicspath{{images/}}
\usepackage{setspace}

% Caption customization
% use this package to set appearance for captions
\usepackage{caption}
% caption size for figures 10pt
\captionsetup[figure]{font=footnotesize,labelfont=footnotesize}
% caption size for tables 10pt and underlined
\usepackage[normalem]{ulem} % Package for underlining
\DeclareCaptionLabelFormat{label_format}{\uline{#1~#2}} % underline label
\DeclareCaptionTextFormat{text_format}{\uline{#1}} % underline text
\DeclareCaptionLabelSeparator{separator_format}{\uline{:~}} % underline separator
\captionsetup[table]{font=normalsize,labelfont=normalsize,labelformat=label_format,textformat=text_format,labelseparator=separator_format}

% use this package to define custom colors
\usepackage{xcolor}

% create colors
\colorlet{punct}{red!60!black}
\definecolor{background}{HTML}{EEEEEE}
\definecolor{delim}{RGB}{20,105,176}
\colorlet{numb}{magenta!60!black}


\usepackage{amsfonts}
\usepackage{amscd}
\usepackage{amssymb}
\newtheorem{algor}{\bf{Algorithm}}[subsection]


\newtheorem{remark}{Remark}[section]

\newtheorem{theorem}{Theorem}[section]
\newtheorem{lemma}[theorem]{Lemma}
%\newtheorem{corollary}[theorem]{Corollary}
\newtheorem{definition}[theorem]{Definition}
\newtheorem{proposition}[theorem]{Proposition}
%\theoremstyle{remark}
%\newtheorem{example}[theorem]{Example}
%\newtheorem{remark}[theorem]{Remark}
\numberwithin{equation}{section}

% use this package to show actual code listings
\usepackage{listings}

% change listings name in caption to Απεικόνιση
\renewcommand{\lstlistingname}{Απεικόνιση}

% change listings name in contents page to Κατάλογος απεικονήσεων
\renewcommand\lstlistlistingname{Κατάλογος απεικονίσεων}

% set custom colorscheme for listings with language=lang1 to make them stand out more
% http://tex.stackexchange.com/questions/83085/how-to-improve-listings-display-of-json-files
\lstdefinelanguage{lang1}{
	basicstyle=\normalfont\ttfamily,
	%    numbers=left,
	%    numberstyle=\scriptsize,
	%    stepnumber=1,
	%    numbersep=8pt,
	showstringspaces=false,
	breaklines=true,
	frame=lines,
	backgroundcolor=\color{background},
	literate=
	*{0}{{{\color{numb}0}}}{1}
	{1}{{{\color{numb}1}}}{1}
	{2}{{{\color{numb}2}}}{1}
	{3}{{{\color{numb}3}}}{1}
	{4}{{{\color{numb}4}}}{1}
	{5}{{{\color{numb}5}}}{1}
	{6}{{{\color{numb}6}}}{1}
	{7}{{{\color{numb}7}}}{1}
	{8}{{{\color{numb}8}}}{1}
	{9}{{{\color{numb}9}}}{1}
	{:}{{{\color{punct}{:}}}}{1}
	{,}{{{\color{punct}{,}}}}{1}
	{\{}{{{\color{delim}{\{}}}}{1}
	{\}}{{{\color{delim}{\}}}}}{1}
	{[}{{{\color{delim}{[}}}}{1}
	{]}{{{\color{delim}{]}}}}{1},
}


% create command for blank page
\usepackage{afterpage}
\newcommand\blankpage{%
	\null
	\thispagestyle{empty}%
	\addtocounter{page}{-1}%
	\newpage}

\definecolor{maroon}{HTML}{AF3235}
% add clickable hyperlinks
\usepackage{hyperref}
\hypersetup{
	colorlinks,
	citecolor=black,
	filecolor=black,
	linkcolor=black,
	urlcolor=black
}

% use fancy header and footer
\usepackage{fancyhdr}
\usepackage{blindtext} % to quickly get a full document

% Turn on the style
\pagestyle{fancy}

% Clear the header and footer
\fancyhf{}

% Set the right side of the footer to be the page number
\fancyfoot[R]{\thepage}

% set page number appearance to bottom right
\fancypagestyle{plain}{%
	\renewcommand{\headrulewidth}{0pt}
	\fancyhf{}
	\fancyfoot[R]{\thepage}%
}

\newcommand{\HRule}{\rule{\linewidth}{0.5mm}}
\newcommand{\lt}{\latintext}

\begin{document}
	\begin{titlepage}
		
		{\LARGE Αριστοτέλειο Πανεπιστήμιο Θεσσαλονίκης}
		\begin{center} {\Large Σχολή Θετικών Επιστημών} \end{center}
		\begin{figure}[h]
			\raggedright
			\hspace{90pt}
			\includegraphics[width=0.5\linewidth]{logo.png}
		\end{figure}
		\begin{center}
			
			
			\begin{figure}[h]
				\centering 
				%\includegraphics[width=0.3\linewidth]{authLogo.png}
			\end{figure}
			\begin{center}
				% leave 2 cm from above text
				
				\HRule \\[0.4cm]
				{\huge Εργασία στο μάθημα της Κρυπτογραφίας}
				
				\HRule \\[0.4cm]
			\end{center}
			
			% put this on the bottom
			\vfill
			\begin{doublespacing}
				
				{\LARGE 
					Φτιάκας Σωτήριος ΑΕΜ: 3076\\}
				{\LARGE 
					Μπάρμπας Γρηγόριος ΑΕΜ: 3108\\}
				
				
				\vfill 
				{\Large \today}
			\end{doublespacing}
		\end{center}
	\end{titlepage}
	
	
	% insert table of contents
	\tableofcontents
	
	\clearpage
	
	
	\addcontentsline{toc}{section}{Περίληψη}
	\section*{{\color{maroon}Περίληψη}}
	
	..............
	% leave blank page before main part
	\blankpage
	
	\addcontentsline{toc}{section}{Θέμα 1}
	\section*{{\color{maroon}Θέμα 1}}
	
	\addcontentsline{toc}{section}{Θέμα 2}
	\section*{{\color{maroon}Θέμα 2}}
	
	\addcontentsline{toc}{section}{Θέμα 3}
	\section*{{\color{maroon}Θέμα 3}}
	
	\section*{\lt{Vigenere}}
	{\lt{
		$$No3\_Vigenere.ipynb$$
	}}
	
	\addcontentsline{toc}{section}{Θέμα 4}
	\section*{{\color{maroon}Θέμα 4}}
	
	\addcontentsline{toc}{section}{Θέμα 5}
	\section*{{\color{maroon}Θέμα 5}}
	
	\section*{\lt{Dictionary Attack}}
	{\lt{
			$$No5\_Dictionary Attack.ipynb$$
	}}
	
	\addcontentsline{toc}{section}{Θέμα 6}
	\section*{{\color{maroon}Θέμα 6}}
	
	\addcontentsline{toc}{section}{Θέμα 7}
	\section*{{\color{maroon}Θέμα 7}}
	
	\section*{\lt{Shift Operator with XOR}}
	
	{\lt{
			m: 16-bits
			$$c=m\oplus(m<<6)\oplus(m<<10)$$
				
			Όπου $m<<a$ είναι κύλιση προς τα αριστερά κατά $a$-bits.
			
			Για μήνυμα $m$ και κλειδί $k$ ισχύει:
			Αν $c = m \oplus k$, τότε $m = c \oplus k$ 
				
			Επιπλέον, στην αρχική μας συνάρτηση κρυπτογράφησης, μπορούμε να κυλίσουμε και τα δύο μέλη ταυτόχρονα.
			$$(c<<2) = (m\oplus(m<<6)\oplus(m<<10))<<2$$
			
			$$\Leftrightarrow (c<<2) = (m<<2)\oplus(m<<8)\oplus(m<<12)$$
				
			Σημείωση: το $x<<i$ θα συμβολίζεται ως $x_i$ για ευκολία. \newline
			Συνεπώς θα έχουμε:	
			\begin{gather*} 
			c_0 = m_0 \oplus m_6 \oplus m_{10} \quad (1) \\
			c_2 = m_2 \oplus m_8 \oplus m_{12} \Rightarrow m_8 = m_2 \oplus m_{12} \oplus c_2 \quad (4) \\
			c_4 = m_4 \oplus m_{10} \oplus m_{14} \Rightarrow m_{10} = m_4 \oplus m_{14} \oplus c_4 \quad (2) \\
			c_6 = m_6 \oplus m_{12} \oplus m_0 \quad (5) \\
			c_8 = m_8 \oplus m_{14} \oplus m_2 \\
			c_{10} = m_{10} \oplus m_0 \oplus m_4 \\
			c_{12} = m_{12} \oplus m_2 \oplus m_6 \\
			c_{14} = m_{14} \oplus m_4 \oplus m_8 \Rightarrow m_{14} \oplus m_4 = m_8 \oplus c_{14} \quad (3) \\
			\end{gather*}
			
			Ξεκινώντας από την (1) έχουμε διαδοχικά:
			\begin{gather*} 
			c_0 = m_0 \oplus m_6 \oplus m_{10} \\ 
			(2)\Rightarrow c_0 = m_0 \oplus m_6 \oplus m_4 \oplus m_{14} \oplus c_4 \\
			(3)\Rightarrow c_0 \oplus c_4 = m_0 \oplus m_6 \oplus m_8 \oplus c_{14} \\
			(4)\Rightarrow c_0 \oplus c_4 \oplus c_{14} = m_0 \oplus m_6 \oplus m_2 \oplus m_{12} \oplus c_2 \\
			(5)\Rightarrow c_0 \oplus c_4 \oplus c_{14} \oplus c_2 = m_2 \oplus c_6 \\
			\Rightarrow c_0 \oplus c_4 \oplus c_{14} \oplus c_2 \oplus c_6 = m_2 \quad (6)
			\end{gather*}
			
			Κάνουμε κύλιση και στα δύο μέρη του (6) προς τα δεξιά και έχουμε:
			$$m_0 = c_{14} \oplus c_2 \oplus c_{12} \oplus c_0 \oplus c_4$$
			
			και άρα τελικά έχουμε: $$m_0 = c_0 \oplus c_2 \oplus c_4 \oplus c_{12} \oplus c_{14}$$
	}}
	Κώδικας σε \lt{python}
	{\lt{
			$$No7\_Shift\_Operator.ipynb$$
	}}
	
	\addcontentsline{toc}{section}{Θέμα 8}
	\section*{{\color{maroon}Θέμα 8}}
	
	\addcontentsline{toc}{section}{Θέμα 9}
	\section*{{\color{maroon}Θέμα 9}}
	
	\section*{\lt{Entropy}}
	{\lt{
		\begin{table}[h!]
			\centering
			\begin{tabular}{|c|c|c|c|c|c|c|c|c|}
				\hline
				\textbf{Υ/X} & \textbf{0} & \textbf{1} & \textbf{2}\\
				\hline
				\textbf{0} & 1/7 & 1/7 & 1/7 \\
				\hline
				\textbf{1} & 0 & 1/7 & 1/7 \\
				\hline
				\textbf{2} & 2/7 & 0 & 0 \\
				\hline
			\end{tabular}
		\end{table}
		Αρχικά υπολογίζουμε
		\begin{gather*} 
			p_x(X=0) = \sum_{y}p_{X,Y}(0,y)=\frac{3}{7} \\
			p_x(X=1) = \sum_{y}p_{X,Y}(1,y)=\frac{2}{7} \\
			p_x(X=2) = \sum_{y}p_{X,Y}(2,y)=\frac{2}{7} \\
			p_y(Y=0) = \sum_{y}p_{X,Y}(x,0)=\frac{3}{7} \\
			p_y(Y=1) = \sum_{y}p_{X,Y}(x,1)=\frac{2}{7} \\
			p_y(Y=2) = \sum_{y}p_{X,Y}(x,2)=\frac{2}{7}
		\end{gather*}
		
		Ισχύει ότι:
		$$H(X) = -\sum_{x}p_{X}(x)\log _{2}p_{X}(x) $$
		
		Επομένως
		\begin{gather*} 
			H(X) = -\frac{3}{7}\log _{2}\frac{3}{7} - \frac{2}{7}\log _{2}\frac{2}{7} - \frac{2}{7}\log _{2}\frac{2}{7} \\
			= -\frac{3}{7}\log _{2}\frac{3}{7} -\frac{4}{7}\log _{2}\frac{2}{7} \\
			\simeq 1.5566567074628228\\
			\\		
			Υ(Χ) = -\frac{3}{7}\log _{2}\frac{3}{7} - \frac{2}{7}\log _{2}\frac{2}{7} - \frac{2}{7}\log _{2}\frac{2}{7} \\
			= -\frac{3}{7}\log _{2}\frac{3}{7} -\frac{4}{7}\log _{2}\frac{2}{7} \\
			\simeq 1.5566567074628228
		\end{gather*}
		
		Επίσης έχουμε τον εξής τύπο
		$$H(X,Y)=-\sum_{x}\sum_{y}p(x,y)\log_{2}p(x,y)$$
		Άρα θα έχουμε:
		\begin{gather*} 
			Η(X,Υ) = - p(0,0)\log_{2}p(0,0) - p(0,1)\log_{2}p(0,1) - p(0,2)\log_{2}p(0,2) - \\ p(1,0)\log_{2}p(1,0) - p(1,1)\log_{2}p(1,1) - p(1,2)\log_{2}p(1,2) - \\
			p(2,0)\log_{2}p(2,0) -  p(2,1)\log_{2}p(2,1) - p(2,2)\log_{2}p(2,2) \\
			\simeq 2.5216406363433186
		\end{gather*}
		
		Θα υπολογίσουμε την $H(Y|X)$.
		Χρειαζόμαστε αρχικά τα παρακάτω,
		\begin{gather*} 
			p_{Y|X}(y=0|x=0) = \frac{p_{X,Y}(0,0)}{p_{X}(0)} = \frac{\frac{1}{7}}{\frac{3}{7}} =\frac{1}{3} \\
			p_{Y|X}(y=1|x=0) = \frac{p_{X,Y}(0,1)}{p_{X}(0)} = \frac{0}{\frac{3}{7}} = 0 \\
			p_{Y|X}(y=2|x=0) = \frac{p_{X,Y}(0,2)}{p_{X}(0)} = \frac{\frac{2}{7}}{\frac{3}{7}} =\frac{2}{3} \\
			p_{Y|X}(y=0|x=1) = \frac{p_{X,Y}(1,0)}{p_{X}(1)} = \frac{\frac{1}{7}}{\frac{2}{7}} =\frac{1}{2} \\
			p_{Y|X}(y=1|x=1) = \frac{p_{X,Y}(1,1)}{p_{X}(1)} = \frac{\frac{1}{7}}{\frac{2}{7}} =\frac{1}{2} \\
			p_{Y|X}(y=2|x=1) = \frac{p_{X,Y}(1,2)}{p_{X}(1)} = \frac{0}{\frac{2}{7}} = 0 \\
			p_{Y|X}(y=0|x=2) = \frac{p_{X,Y}(2,0)}{p_{X}(2)} = \frac{\frac{1}{7}}{\frac{2}{7}} =\frac{1}{2} \\
			p_{Y|X}(y=1|x=2) = \frac{p_{X,Y}(2,1)}{p_{X}(2)} = \frac{\frac{1}{7}}{\frac{2}{7}} =\frac{1}{2} \\
			p_{Y|X}(y=2|x=2) = \frac{p_{X,Y}(2,2)}{p_{X}(2)} = \frac{0}{\frac{2}{7}} = 0
		\end{gather*}		
	}}

	Τώρα πρέπει να υπολογίσουμε τα παρακάτω: 
	\begin{gather*} 
		H(Y|X=0) = - \sum_{y}p_{Y|X}(y|x=0)\log_{2}p_{Y|X}(y|x=0) \\
		= -(\frac{1}{3}\log_{2}\frac{1}{3} + 0 + \frac{2}{3}\log_{2}\frac{2}{3}) \\
		= -(\frac{1}{3}\log_{2}\frac{1}{3} + \frac{2}{3}\log_{2}\frac{2}{3}) \\
		H(Y|X=1) = - \sum_{y}p_{Y|X}(y|x=1)\log_{2}p_{Y|X}(y|x=1) \\
		= -(\frac{1}{2}\log_{2}\frac{1}{2} + \frac{1}{2}\log_{2}\frac{1}{2} + 0) \\
		= - \log_{2}2 = 1\\
		H(Y|X=2) = - \sum_{y}p_{Y|X}(y|x=2)\log_{2}p_{Y|X}(y|x=2) \\
		= - \log_{2}2 = 1\\
	\end{gather*}
	
	Τότε θα έχουμε
	\begin{gather*} 
		H(Y|X) = \sum_{x}p_{X}xH(Y|X=x) \\
		= p_{X}(0)H(Y|X=0) + p_{X}(1)H(Y|X=1) + p_{x}(2)H(Y|X=2) \\
		\simeq 0.9649839288804954
	\end{gather*}
	
	Γνωρίζουμε επίσης ότι από το θεώρημα της αμοιβαίας πληροφορίας έχουμε
	$$I(X,Y) = H(X) - H(X|Y) = H(Y) - H(Y|X)$$
	
	Άρα έχουμε
	\begin{gather*} 
		H(X|Y) = -(H(Y) - H(Y|X) - H(X)) \\
		\simeq 0.9649839288804954 
	\end{gather*}
	
	Τέλος,
	\begin{gather*} 
		ρ = 1 - \frac{H(Y|X)}{H(X)} \\
		\simeq 0.5916727785823274
	\end{gather*}
	
	Κώδικας σε \lt{python}
	{\lt{
			$$No9\_Entropy.ipynb$$
	}}
	
	\addcontentsline{toc}{section}{Θέμα 10}
	\section*{{\color{maroon}Θέμα 10}}
	
	\addcontentsline{toc}{section}{Θέμα 11}
	\section*{{\color{maroon}Θέμα 11}}
	
	\addcontentsline{toc}{section}{Θέμα 12}
	\section*{{\color{maroon}Θέμα 12}}
	
	\addcontentsline{toc}{section}{Θέμα 13}
	\section*{{\color{maroon}Θέμα 13}}
	
	\section*{\lt{Chinese Theorem}}
	{\lt{
			Έχουμε το σύστημα των γραμμικών ισοδυναμιών
			\begin{gather*} 
				x \equiv 9 \pmod{19} \\
				x \equiv 9 \pmod{12} \\
				x \equiv 13 \pmod{19}
			\end{gather*}
			
			Έχουμε ότι ισχύει $\gcd(12,17,19) = 1$, άρα δεν απαιτείται κάποια απλοποίηση.
			
			Για την επίλυση του συστήματος χρησιμοποιούμε το Κινέζικο Θεώρημα Υπολοίπων
			
			Έτσι έχουμε: $m=17*12*19=3876$
			
			\begin{gather*} 
				M_1 = 228y_1 \equiv 1 \mod{17} \implies 7y_1 \equiv \mod{17} \implies y_1=5 \\
				M_2 = 323y_2 \equiv 1 \mod{12} \implies 11y_1 \equiv \mod{12} \implies y_1=11 \\
				M_3 = 204y_3 \equiv 1 \mod{19} \implies 14y_1 \equiv \mod{19} \implies y_1=15
			\end{gather*}
			
			Τώρα πολλαπλασιάζουμε και προσθέτουμε:
			\begin{gather*}
				x = 9*228*5 + 9*323*11 + 13*204*15 \\
				= 82017 (1)
			\end{gather*}
			Παρατηρούμε ότι η (1) γράφεται,
			$$x = 82017 = 621 + 3876k, k \in \mathbb{Z} $$
			
			Για $k=0$ έχουμε λύση το $x=621$
	}}

	Κώδικας σε \lt{python}
	{\lt{
			$$No13\_ChineseTheorem.ipynb$$
	}}
	
	\addcontentsline{toc}{section}{Θέμα 14}
	\section*{{\color{maroon}Θέμα 14}}
	
	\addcontentsline{toc}{section}{Θέμα 15}
	\section*{{\color{maroon}Θέμα 15}}
	
	\addcontentsline{toc}{section}{Θέμα 16}
	\section*{{\color{maroon}Θέμα 16}}
	
	\section*{\lt{GPG, PGP, Send Message}}
	{\lt{
		Μπάρμπας Γρηγόριος: $$b641ed06419f8ff4a6447cc9fb9d2295$$
		
		Φτιάκας Σωτήριος: $$5f816b4f295dc95721a7a34b9fd1653a$$
	}}
	
	\addcontentsline{toc}{section}{Θέμα 17}
	\section*{{\color{maroon}Θέμα 17}}
	
	\addcontentsline{toc}{section}{Θέμα 18}
	\section*{{\color{maroon}Θέμα 18}}
	
	\section*{\lt{secure.zip}}
	{\lt{
			$$9e94b15ed312fa42232fd87a55db0d39$$
	}}
	
	\addcontentsline{toc}{section}{Θέμα 19}
	\section*{{\color{maroon}Θέμα 19}}
	
	\addcontentsline{toc}{section}{Θέμα 20}
	\section*{{\color{maroon}Θέμα 20}}
	
	\addcontentsline{toc}{section}{Θέμα 21}
	\section*{{\color{maroon}Θέμα 21}}
	
	\addcontentsline{toc}{section}{Θέμα 22}
	\section*{{\color{maroon}Θέμα 22}}
	
	\section*{\lt{3.1}}
	{\lt{
			Αρχικά θέλουμε να αποδείξουμε ότι οι αριθμοί της μορφής $4n+3$ δεν είναι τέλεια τετράγωνα.
			Αρχικά για $n\leq0$ εύκολα παρατηρούμε ότι ισχύει η παραπάνω πρόταση. Αυτό συμβαίνει καθώς για $n=0$ έχουμε το $3$ το οποίο δεν είναι τέλειο τετράγωνο και για $n < 0$ το $4n+3$ είναι αρνητικός. 
			
			Έστω ότι,
			$$4n+3=a^2, \quad n,a \in \mathbb{N}^* (1)$$
			
			Εφόσον $a\in \mathbb{N}^*$, τότε μπορούμε να πούμε ότι $a=2k+1, \quad k \in \mathbb{N}$.
			
			Αντικαθιστώντας το $a$ στην (1) έχουμε:
			\begin{gather*}
			4n+3 = (2k+1)^2 \implies 4n+3 = 4k^2 + 4k + 1 \\
			\equiv 4k^2 + 4k - 4n = 2 \\
			\equiv 2(k^2 + k - n) = 1 \\
			\equiv k^2 + k - n = \frac{1}{2}
			\end{gather*}
			
			Το οποίο είναι άτοπο καθώς $k,n \in \mathbb{N}^*$ και άρα το $k^2\in \mathbb{N}^* $
			αλλά και όλη η παράσταση $k^2 + k - n \in \mathbb{N}$, εφόσον είναι άθροισμα των φυσικών αριθμών $k^2, k$ και $-n$.
			
			Συνεπώς και η αρχική ισοδύναμη υπόθεση είναι άτοπη, οπότε το $4n+3$ δεν είναι τέλειο τετράγωνο.
			
			Για το δεύτερο ερώτημα παρατηρούμε ότι όλοι αριθμοί
			$$11,111,...,111\cdotp\cdotp\cdotp111,...$$ 
			μπορούν να γραφούν στην μορφής $(4n+3) + 10^q$, συνεπώς αυτό το σύνολο αριθμών δεν θα έχει τέλειο τετράγωνο.
	}}
	
	\addcontentsline{toc}{section}{Θέμα 23}
	\section*{{\color{maroon}Θέμα 23}}
	
	\section*{\lt{3.4}}
	{\lt{
			Υπάρχουν δύο περιπτώσεις που θα χρειαστεί να εξετάσουμε.
			
			Περίπτωση 1:
			Για περιττό αριθμό διαδοχικών αριθμών, αυτοί οι αριθμοί θα έχουν ως μέσο έναν ακέραιο (μεσσαίος αριθμός), οπότε το άθροισμα γράφεται ως εξής:
			$$sum = average * number\_of\_consecutive\_numbers$$
			$$\implies sum = integer * odd\_number$$
			Αυτό σημαίνει ότι το άθροισμα ($sum$) διαιρείται από έναν περιττό αριθμό. Αυτό όμως δεν μπορεί να είναι το σενάριο για το $2^m$.
			
			Περίπτωση 2:
			Ένας ζυγός αριθμός διαδοχικών αριθμών έχουν ως μέσο το μέσο του αθροίσματος των δύο μεσαίων. Συνεπώς έχουμε:
			$$sum = ((sum\_of\_two\_middle\_numbers) * \frac{1}{2}) * number\_of\_consecutive\_numbers$$
			$$\implies sum = (sum\_of\_two\_middle\_numbers) * \frac{1}{2} * even\_number$$
			$$\implies sum = (sum\_of\_two\_middle\_numbers) * integer$$
			$$\implies sum = ((k) + (k+1)) * integer \quad , k \in \mathbb{Z}$$
			$$\implies sum = (2k+1) * integer$$
			Το $2k+1$ είναι περιττός αριθμός, άρα το άθροισμα ($sum$) έχει ως παράγοντα περιττό, άρα όπως και προηγουμένως απορρίπτεται το σενάριο $2^m$.
	}}
	
	\addcontentsline{toc}{section}{Θέμα 24}
	\section*{{\color{maroon}Θέμα 24}}
	
	\addcontentsline{toc}{section}{Θέμα 25}
	\section*{{\color{maroon}Θέμα 25}}
	
	\addcontentsline{toc}{section}{Θέμα 26}
	\section*{{\color{maroon}Θέμα 26}}
	
	\addcontentsline{toc}{section}{Θέμα 27}
	\section*{{\color{maroon}Θέμα 27}}
	
	\section*{\lt{3.26}}
	{\lt{
			\begin{enumerate}[label=(\roman*)]
				\item Έστω ότι $d_1 = \gcd(c,b)$ και $d_2 = \gcd(ac,b)$
				
				Τότε έχουμε $c \cdot x_1+b \cdot y_1 = d_1$, $a \cdot c \cdot x_2+b \cdot y_2 = d_2$ και $a \cdot x+b \cdot y=1$, από Bezout.
				Αρχικά πολλαπλασιάζουμε την $a \cdot x+b \cdot y=1$ με το $d_1$ και έχουμε:
					\begin{gather*}
					d_1 \cdot (a \cdot x+b \cdot y)=1 \cdot d_1 \\
					\implies a \cdot x(c \cdot x_1+b \cdot y_1)+b \cdot d_1 \cdot y = d_1 \\
					\implies a \cdot c \cdot (x \cdot x_1)+b \cdot (a \cdot x \cdot y_1+d_1 \cdot y) = d_1
					\end{gather*}
				Εφόσον ισχύει ότι $d_2 = \gcd(ac,b)$, τότε διαιρεί κάθε ακέραιο γραμμικό συνδυασμό των $ac$ και $b$ και άρα έχουμε $d_2|d_1 \quad(1)$.
				
				Στην συνέχεια θα πολλαπλασιάσουμε ομοίως το $a \cdot x+b \cdot y=1$ με το $d_2$ και έχουμε:
					\begin{gather*}
					d_2 \cdot (a \cdot x+b \cdot y)=1 \cdot d_2 \\
					\implies a \cdot x(a \cdot c \cdot x_2+b \cdot y_2)+b \cdot d_2 \cdot y = d_2 \\
					\implies c \cdot (a^2 \cdot x \cdot x_2) + b \cdot (a \cdot x \cdot y_2 + d_2 \cdot y) = d_2
					\end{gather*}
				Ομοίως με προηγουμένως ισχύει ότι $d_1 = \gcd(c,b)$, τότε διαιρεί κάθε ακέραιο γραμμικό συνδυασμό των $ac$ και $b$ και άρα έχουμε $d_1|d_2 \quad(2)$.
				Από (1) και (2) έχουμε ότι $d_1 = d_2$, μη αρνητικά.
				
				Αφού 
				$$(1) \implies |d_1| \leqslant |d_2|$$
				$$(2) \implies |d_2| \leqslant |d_1|$$
				\item Έστω $d$ κοινός διαρέτης των $a+b$ και $a-b$, τότε ο $d$ διαιρεί και το άθροισμα και την διαφορά τους.
				$$d|(a+b)$$
				$$d|(a-b)$$
				$$d|(a+b)+(a+b) = 2 \cdot a$$
				$$d|(a+b)-(a-b) = 2 \cdot b$$
				
				Τότε έχουμε ότι:
				$$d|\gcd(2a,2b)=2\gcd(a,b)$$
				Όμως από αρχικά δεδομένα έχουμε ότι $\gcd(a,b)=1$ άρα τότε ισχύει:
				$$d|2$$
				Συνεπώς το $d \in \{1,2\}$
				
				Εαν a,b περιττοί τότε έχουμε το εξής:
				$$a = 2k_1+1 \quad k_1 \in \mathbb{Z}$$
				$$b = 2k_2+1 \quad k_2 \in \mathbb{Z}$$	
				Άρα θα ισχύει και το εξής:
				$$a+b = 2k_1+1 + 2k_2+1 = 2k_1 + 2k_2 + 2 = 2\cdot (k_1+k_2+1) \quad (even)$$
				$$a-b = 2k_1+1 - 2k_2+1 = 2k_1 - 2k_2 = 2\cdot (k_1 - k_2) \quad (even)$$
				Εφόσον και τα δύο είναι ζυγοί αριθμοί τότε οι διαιρέτες τους θα είναι ζυγοί.
				Όπως αποδείξαμε προηγούμενος όμως, αν $d$ διαιρέτης, τότε $d \in \{1,2\}$
				Συμπερασματικά έχουμε ότι $d=2$.
				
				\item Έστω $d=\gcd(a,b)$ τότε $d=a\cdot x + b\cdot y, \quad x,y \in \mathbb{Z}$
				Αν $i=\gcd(2^a-1, 2^b-1)$ τότε
				$$2^a \equiv 1 \mod{i}$$
				$$2^b \equiv 1 \mod{i}$$
				Συνεπώς έχουμε
				$$2^d = 2^{a\cdot x + b\cdot y} = (2^a)^x \cdot (2^b)^y \equiv 1 \mod{i}$$
				Οπότε $p|2^d-1$.
				Από την άλλη μεριά αν $d|a$, τότε $2^d-1|2^q-1$, οπότε το $2^d-1$ αποτελεί κοινό παράγοντα.
				Έτσι αποδείξαμε ότι 
				$$\gcd(2^a-1,2^b-1)=2^{\gcd(a,b)}-1$$
				Όμως από αρχικά δεδομένα $d=1$, άρα και η προηγούμενη σχέση γράφεται:
				$$\gcd(2^a-1,2^b-1)=1$$
				\item Παρατηρούμε ότι αν αντικαταστήσουμε τα $M_p, M_q$ έχουμε την παραπάνω σχέση.
			\end{enumerate}
	}}
	
	\addcontentsline{toc}{section}{Θέμα 28}
	\section*{{\color{maroon}Θέμα 28}}
	
	\addcontentsline{toc}{section}{Θέμα 29}
	\section*{{\color{maroon}Θέμα 29}}
	
	\addcontentsline{toc}{section}{Θέμα 30}
	\section*{{\color{maroon}Θέμα 30}}
	
	\section*{\lt{3.70}}
	{\lt{
			Υπόθεση:\\
			$
			N>2 \\
			N = p_1p_2\ldots p_k \\
			p_j-1|N-1 \forall j
			$\\
			Απόδειξη:\\
			Έστω $\gcd(a,N)=1$. \\
			Από το θεώρημα του Fermat, $\forall j$, έχουμε $a^{p_j} \equiv 1 \mod{p_j}$.\\
			Εφόσον $p_j-1|N-1$,και άρα $a^{N-1}\equiv 1 \mod{p_j}$. \\
			Δηλαδή το $a^{N-1}-1$ είναι πολλαπλάσιο κάθε $p_j$. \\
			Συνεπώς $a^{N-1}\equiv 1 \mod{Ν}$.
	}}
	
	\addcontentsline{toc}{section}{Θέμα 31}
	\section*{{\color{maroon}Θέμα 31}}
	
	\section*{\lt{3.74}}
	{\lt{
			Παρατηρούμε ότι $561(=3\cdot 11 \cdot 17)$ είναι αριθμός Carmichael. Θα βρούμε όλους του αριθμούς Carmichael μέχρι $Ν(=3000)$.
			
			Πρόταση 1:
			Έστω $n=p \cdot u$ όπου $p$ είναι πρώτος. Τότε αν και μόνο αν $p-1|u-1$ θα ισχύει και $p-1|n-1$.
			$$(n-1)-(u-1) = n-u = p\cdot u-u=(p-1)\cdot u$$
			
			Πρόταση 2:
			Έστω ένας αριθμός Carmichael έχει τουλάχιστον τρεις πρώτους παράγοντες. Για την απόδειξη αυτής της πρότασης εφαρμόζουμε την εξής λογική:
			
			Έστω ότι ο $n$ έχει δύο πρώτους παράγοντες $n=p\cdot q$ όπου $p,q$ πρώτοι και $p>q$. Τότε $p-1 > q-1$, άρα το $p-1$ δεν διαιρεί το $q-1$. Από την πρόταση (1) το $p-1$ δεν διαιρεί το $n-1$. Συνεπώς το $n$ δεν είναι αριθμός Carmichael.
			
			Πρόταση 3:
			Ας υποθέσουμε ότι ο $n$ είναι Carmicael και ότι το $p$ και το $q$ είναι πρώτοι παράγοντες του n. Τότε $q\not\equiv 1 \mod p$.
			
			Έστω ότι το $q\equiv 1 \mod p$, έτσι ισχύει ότι $p|q-1$. Τότε $q-1|n-1$ καθώς θα ισχύει και ότι $p|n-1$. Όμως αυτό είναι άτοπο καθώς ισχύει ότι $p|n$.
			
			Εύρεση αριθμών Carmichael:
			Έστω αριθμός $n$ με τρεις πρώτους παράγοντες $n=p\cdot q\cdot r$ , με $p<q<r$. Από τα προηγούμενα καταλαβαίνουμε ότι χρειαζόμαστε τριπλέτες $(p,q,r)$ για τις οποίες θα ισχύουν τα εξής:
			
			\begin{gather*}
			(i) \quad p-1|q\cdot r-1 \quad (or \quad q\cdot r \equiv 1 \mod (p-1)) \\ 
			(ii) \quad q-1|p\cdot r-1 \\ 
			(iii) \quad r-1|p\cdot q-1 
			\end{gather*}
			
			Δοθεί ένα ζευγάρι πρώτον αριθμών $(p,q)$ με $p<q$, η ακόλουθη διαδικασία θα εντωπίσει όλους τους πρώτους $r>q$ τέτοιοι ώστε το $p\cdot q\cdot r$ να είναι αριθμός Carmichael.
			
			Έστω οι ζυγοί διαρέτες (αν υπάρχουν) $d$ του $p\cdot q -1$ με $p<d<p\cdot q -1$ και ελέγχουμε αν $d+1 (=r)$ είναι πρώτος, εξαιρούμε το $d=p\cdot q-1$ καθώς θα μας έδινε $r=p\cdot q$. Τότε έχουμε εξασφαλίσει το (iii) και ελέγχουμε λοιπόν αν ισχύουν τα (ii) και (ii).
			
			Το κάνουμε για όλα τα ζευγάρια πρώτων $(p,q)$, όπου $p\cdot q\cdot r < 3000$, για πρώτους $r>q$. Όμως λόγω του (3) αφήνουμε εκτός τους συνδυασμούς για τους οποίους ισχύει: $q \equiv 1 \mod p$ (π.χ $(3,7)$).
			
			Καταγράφουμε μόνο τις τιμές του $d$ όπου το $r$ είναι πρώτος.
			Όπως παρατηρούμε και στον πίνακα από κάτω δεν υπάρχει μικρότερος  Carmichael με τρείς παράγοντες από τον 561.
			\begin{center}
				\begin{tabular}{||c c c c c c c||} 
					\hline
					$(p,q)$ & $p\cdot q-1$ & $d$ & $r$  & (i) & (ii)  & Carmichael \\ [0.5ex] 
					\hline\hline
					$(3,5)$ & $14$ & $-$ & $-$ &  &  &  \\ 
					\hline
					$(3,11)$ & $32$ & $16$ & $17$ & yes & yes & $3\cdot 11 \cdot 17 = 561$ \\
					\hline
					$(3,17)$ & $50$ & $-$ & $-$ &  &  &  \\
					\hline
					$(3,23)$ & $68$ & $-$ & $-$ &  &  &  \\
					\hline
					$(5,7)$ & $34$ & $-$ & $-$ &  &  &  \\ [1ex] 
					\hline
					$(5,13)$ & $64$ & $16$ & $17$ & yes & yes & $5\cdot 13 \cdot 17 = 1105$ \\ [1ex] 
					\hline
					$(5,17)$ & $84$ & $28$ & $29$ & yes & yes & $5\cdot 17 \cdot 29 = 2465$ \\ [1ex] 
					\hline
					$ $ & $ $ & $42$ & $43$ & no &  &  \\ [1ex] 
					\hline
					$(5,19)$ & $94$ & $-$ & $-$ &  &  &  \\ [1ex] 
					\hline
					$(7,11)$ & $76$ & $-$ & $-$ &  &  &  \\ [1ex] 
					\hline
					$(7,13)$ & $90$ & $18$ & $19$ & yes & yes & $7\cdot 13 \cdot 19 = 1729$ \\ [1ex] 
					\hline
					$ $ & $ $ & $30$ & $31$ & yes & yes & $7\cdot 13 \cdot 31 = 2821$ \\ [1ex] 
					\hline
					$(7,17)$ & $118$ & $-$ & $-$ &  &  &  \\ [1ex] 
					\hline
					$(11,13)$ & $142$ & $-$ & $-$ &  &  &  \\ [1ex] 
					\hline
					
				\end{tabular}
			\end{center}
			
			Το δεύτερο ερώτημα επιλύθηκε με βοήθεια κώδικα.
			
			Κώδικας σε \lt{python}
			{\lt{
					$$No31\_Smaller\_Carmichael\_4\_Factors.ipynb$$
			}}
	}}
	
	\bibliographystyle{plain}
	\bibliography{bib.bib}
	
\end{document} 
